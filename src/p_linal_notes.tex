\documentclass[12pt]{article}

\usepackage[utf8]{inputenc}
\usepackage[T2A]{fontenc}
\usepackage[russian]{babel}
\usepackage{amsmath, amssymb, amsthm}
\usepackage{geometry}
\usepackage{titlesec}
\usepackage{enumitem}
\usepackage[dvipsnames]{xcolor}
\usepackage{tcolorbox}
\usepackage{hyperref}

% Всякие кастомные команды
\newcommand{\R}{\mathbb{R}}
\newcommand{\N}{\mathbb{N}}
\newcommand{\Z}{\mathbb{Z}}
\newcommand{\Q}{\mathbb{Q}}

% Поля
\geometry{a4paper, margin=2.5cm}

% Цвета
\definecolor{myblue}{RGB}{44, 62, 80}
\definecolor{mygreen}{RGB}{39, 174, 96}
\definecolor{myred}{RGB}{192, 57, 43}

% Заголовки
\titleformat{\section}{\normalfont\Large\bfseries\color{myblue}}{\thesection}{1em}{}
\titleformat{\subsection}{\normalfont\large\bfseries\color{mygreen}}{\thesubsection}{1em}{}

% Гиперссылки
\hypersetup{
    colorlinks=true,
    linkcolor=myblue,
    urlcolor=blue!70!black,
    citecolor=blue!70!black
}

% Теоремы и блоки
\tcbuselibrary{breakable, skins}

\tcbset{
    common/.style={
        fonttitle=\bfseries,
        breakable,
        enhanced,
        boxrule=0.5pt,
        arc=3pt,
        coltitle=black,
        top=4pt,
        bottom=4pt,
        left=4pt,
        right=4pt,
        boxsep=4pt,
        before skip=10pt,
        after skip=10pt,
        attach boxed title to top left={xshift=10pt, yshift=-3pt},
        boxed title style={
            colback=white,
            sharp corners,
            boxrule=0pt,
            underlay={\path[fill=#1] (title.south west) rectangle (title.south east);},
        }
    }
}

% Определение
\newtcolorbox{definitionbox}{
    common=blue!30,
    colback=blue!6,
    colframe=blue!80,
    title=Определение
}

% Теорема
\newtcolorbox{theorembox}{
    common=red!30,
    colback=red!6,
    colframe=red!80,
    title=Теорема
}

% Замечание
\newtcolorbox{remarkbox}{
    common=orange!30,
    colback=orange!6,
    colframe=orange!80,
    title=Замечание
}

% Пример
\newtcolorbox{homeworkbox}{
    common=green!30,
    colback=green!6,
    colframe=green!80,
    title=Домашнее задание
}

% Задача
\newtcolorbox{taskbox}{
    common=violet!30,
    colback=violet!6,
    colframe=violet!80,
    title=Задача
}


% Начало документа
\title{\textbf{\Huge Линейная алгебра}\\}
\author{Илья Панов}
\date{\today}

\begin{document}

\maketitle
\tableofcontents
\newpage

% Привет, добро пожаловать в мой конспект. Если хочешь дополнить его, то пиши, пожалуйста, в моём стиле. Мой конспект поддерживает цветовые блоки:

%\begin{definitionbox}
% Множество называется \textbf{ограниченным}, если существует такое число $M > 0$, что для любого $x \in A$ выполнено $|x| < M$.
%\end{definitionbox}

%\begin{theorembox}
% Если функция $f$ непрерывна на отрезке $[a,b]$ и $f(a)\cdot f(b)<0$, то существует $c \in (a,b)$ такое, что $f(c)=0$.
%\end{theorembox}

%\begin{examplebox}
% Функция $f(x) = x^3 - 2x + 1$ имеет корень на $[0,1]$, так как $f(0)=1$, $f(1)=0$.
%\end{examplebox}

%\begin{itemize}
    % \item $\mathbb{R}$ — множество действительных чисел;
    % \item $\mathbb{N}$ — множество натуральных чисел;
    % \item $\mathbb{Z}$ — множество целых чисел.
%\end{itemize}

% Пример формулы:
%\[
% \int_a^b f(x)\,dx = F(b) - F(a)
%\]

\section{Введение}

Конспект в основном составлялся по \href{https://drive.google.com/drive/folders/1gqBksrkU89tgdOTuxG6GgE3jnozzafNv?usp=sharing}{лекциям} Поступашек для подготовки к поступлению в ШАД и AI Masters. Записи лекций у вас есть (я их пронумеровал, смотрите по порядку), домашки есть в конце каждой секции (курс рекомендует выполнять хотя бы по 10 заданий из дз), учебники и сборники задач - в репозитории.

\section{Аналитическая геометрия}

Сейчас мы начнём с повторения 10-11 класса школы, повыводим всякое для плоскости, потом заметим, что для пространства у нас особо ничего не меняется.\\

\begin{theorembox}
Три вектора (a, b, c) на плоскости всегда линейно зависимы\\

\textbf{Доказательство:}

Можем просто составить систему уравнений, решить её по Гауссу (или как Вам угодно).

\[
\begin{cases}
x \cdot x_a + y \cdot x_b = x_c \\
x \cdot y_a + y \cdot y_b = y_c
\end{cases}
\]

Получим, что решения у нас есть, если вектора не коллинеарны (в конце получим $x_a \cdot y_b \ - x_b \cdot y_a$ в знаменателе, если это выражение равно 0, то это равносильно коллинеарности векторов a и b без ограничения общности). Если какие-то два коллинеарны (а третий не коллинеарен), то колинеарные вектора связаны каким-то коэффициентом k, а третий вектор можем взять с нулевым коэффициентом.
\end{theorembox}

\begin{definitionbox}
    \textbf{Метод Гаусса}, также известный как метод исключения Гаусса, это алгоритм решения систем линейных алгебраических уравнений (СЛАУ) путем последовательного исключения переменных. В основе метода лежит преобразование системы уравнений к равносильной ступенчатой (треугольной) форме (то есть нолики у нас снизу выстраиваются), из которой затем последовательно находятся значения переменных. 
\end{definitionbox}

\begin{theorembox}
Угол между двумя векторами и равносильность определений скалярного произведения\\

\textbf{Доказательство:}

Если есть равносильность определений, то косинус угла выражается очевидно. Равносильность следует из теоремы косинусов: пусть хотим найти угол между векторами a и b, тогда проведем третий - c такой, что он соединяет концы двух других векторов. Пишем теорему косинусов для c, как раз получаем искомый угол и связь определений скалярного.
\end{theorembox}

\begin{theorembox}
Неравенство КБШ: произведение длин векторов не меньше, чем модуль их скалярного произведения: $|a| \cdot |b| \geq |(a, b)|$\\

\textbf{Доказательство:}

Рассмотрим $t \in \R$, теперь возьмем скалярное произведение $(x - ty, x- ty)$, оно $\geq 0$ по свойствам. По линейности раскрываем, получаем квадратный трёхчлен, который $\geq 0$, значит у него $D \geq 0$ - это в точности неравенство КБШ.
\end{theorembox}

\begin{theorembox}
Пусть даны два вектора a и b, отложенные от одной точки, тогда проекция вектора a на вектор b можно найти по формуле $a^` = \frac{(a, b)}{(b, b)} \cdot b$\\

\textbf{Доказательство:}

Что такое проекция, надеюсь, все представляют (просто уронили перпендикуляр). Длина проекция очевидным образом находится из прямоугольного треугольника $|a^`| = |a| \cdot \cos{\phi}$\\

Теперь попробуем выразить сам вектор $a^`$, он лежит на b, тогда чтобы получить вектор проекции, мы хотим использовать направление вектора b (единичный вектор) и умножить получившийся вектор на длину проекции: $a^` = \frac{b}{|b|} \cdot |a^`|$. Подставляем $|a^`|$, $\cos{\phi}$ заменяем на $\frac{(a, b)}{|a| \cdot |b|}$, получили требуемое.

\end{theorembox}

\begin{theorembox}
Точка $(x, y)$ принадлежит прямой $l$ (прямая задана точкой $(x_0, y_0)$ и направляющим вектором $(\alpha, \beta)$) тогда и только тогда, когда $\frac{x - x_0}{\alpha} = \frac{y - y_0}{\beta}$. Это равенство мы будем называть каноническим уравнением прямой. В эту же теорему включим вывод других способов задать прямую\\

\textbf{Доказательство:}

Очев: если у нас точка $(x, y)$ лежит на прямой, тогда у нас вектора $(x - x_0, y - y_0)$ и $(\alpha, \beta)$ колинеарны, тогда $\exists \ k \in \R \ | \ (x - x_0, y - y_0) = k \cdot (\alpha, \beta)$. Рассмотрев это равенство покоординатно, получим требуемое отношение. В обратную сторону аналогично, просто введём $k$, скажем про коллинеарность, дальше принадлежность точки прямой очевидна.\\

Из получившегося уравнения очевным образом получаем параметрическое уравнение прямой:

\[
\begin{cases}
x = x_0 + t \cdot \alpha \\
y = y_0 + t \cdot \beta
\end{cases}
\]

По сути заменили k на t. Далее получим общее уравнение прямой $Ax + By +C = 0$. Просто возьмём каноническое и крест-накрест перемножим. Получим $\beta x - \alpha y - x_o \beta + y_0 \alpha = 0$. Дальше мы просто занимаемся переобозначением.
\end{theorembox}

\begin{remarkbox}
    Также заметим, что вектор с координатами $(A, B)$ (читать как $(\beta, -\alpha)$) - это нормаль-вектор нашей прямой. Проверяется через скалярное произведение (помним, что $(\alpha, \beta)$ - это направляющий вектор нашей прямой).
\end{remarkbox}

\begin{theorembox}
Прямая $l: Ax +By + C = 0$ разбивает плоскость на 2 полуплоскости. Если мы возьмём какие-то 2 точки $I_1, I_2$ из разных полуплоскостей, тогда $l(I_1) \cdot l(I_2) < 0$\\

\textbf{Доказательство:}

Зафиксируем точку $(x_0, y_0) \in l$. Теперь рассмотрим скалярное произведение нормаль-вектора и $(x_1 - x_0, y_1 - y_0)$ (если считать, что у точки $I_1$ координаты $(x_1, y_1)$). Аналогично для второй точки $I_2$. Тогда одно скалярное произведение будет $>0$, а другое $<0$ в силу свойства скалярного (если точнее, то просто пользуемся, что косинус тупого угла отрицательный).\\
Например, подробнее для точки $I_1$ из верхней полуплоскости (БОО): $A(x_1 - x_0) + B(y_1 - y_0) > 0$, раскроем скобки, обозначим $C = -Ax_0 - By_0$. Потом мы всё это перемножим с выражением для второй точки: $Ax_2 + Bx_2 + C < 0$. Получим то, что и хотели: $l(I_1) \cdot l(I_2) < 0$.

\end{theorembox}

\begin{theorembox}
Формула расстояния от точки $(x_0, y_0)$ до прямой $l: Ax + By + C = 0$ - это $d(x_0, y_0) = \frac{|Ax_0 + By_0 + C|}{\sqrt{A^2 + B^2}}$\\

\textbf{Доказательство:}

Рассмотрим точку $(x, y) \in l$. Теперь построим вектор $(x_0 - x, y_0 - y)$, спроецируем его на нормаль вектор (точнее мы хотим посмотреть на длину проекции): $|(A, B)| \cdot |\frac{A(x_0 - x) + B(y_0 - y)}{A^2 + B^2}|$. $|(A, B)|$ - это внезапно $\sqrt{A^2 + B^2}$. Сокращаем, вводим обозначение C, получаем требуемое.
\end{theorembox}

\begin{remarkbox}
Обсудим взаимное расположение прямых: совпадают, параллельны или пересекаются. В терминах коэффициентов это соответственно $\frac{A_1}{A_2} = \frac{B_1}{B_2} = \frac{C_1}{C_2}$, $\frac{A_1}{A_2} = \frac{B_1}{B_2}$ или никакое из предыдущих равенств не выполняется.\\

В терминах матриц совпадение это:

\[
rk \begin{pmatrix}
A_1 & B_1\\
A_2 & B_2
\end{pmatrix}
 = 1\]

\[
rk \begin{pmatrix}
A_1 & B_1 & C_1\\
A_2 & B_2 & C_2
\end{pmatrix}
 = 1\]

Аналогично для параллельности у нас ранг большой матрицы будет 2, а в случае пересечения у нас ранг и маленькой, и большой матрицы будет 2.

\end{remarkbox}

\begin{definitionbox}
    \textbf{Ранг матрицы (rk)} - это максимальный порядок минора матрицы, отличный от нуля. Иными словами, это число, равное максимальному количеству линейно независимых строк (или столбцов) в матрице. Ранг матрицы показывает размерность подпространства, натянутого на строки (или столбцы) матрицы.
\end{definitionbox}

\begin{theorembox}
Площадь параллелограмма, построенного на векторах $(a, c)$ и $(b, d)$ - это определитель:
\[
\begin{vmatrix}
a & c\\
b & d
\end{vmatrix}
\]

\textbf{Доказательство:}

$S = |a| \cdot |b| \cdot \sin{\phi}$. Меняем синус на косинус по ОТТ, заносим всё под корень, раскрываем скобки, там у нас получается полный квадрат: $\sqrt{(ad - bc)^2}$, а это в точности определитель.
\end{theorembox}

\begin{remarkbox}
Из такого геометрического смысла определителя становятся очевидны всякие свойства про линейность по строке, иммутабельность при транспонировании.\\

Такую же формулу, кстати, можно вывести для $\R^3$, но это будет просто более глиномесно:

\[
V = 
\begin{vmatrix}
c_1 & c_2 & c_3\\
a_1 & a_2 & a_3\\
b_1 & b_2 & b_3
\end{vmatrix}
\]
\end{remarkbox}

Вот теперь мы плавно перешли к пространствам. Будем волшебным образом перетаскивать формулы из плоскости, натягивать их на пространство, также будем что-то новое вводить.\\

\begin{remarkbox}
Каноническое уравнение прямой в $\R^3$ - это $\frac{x - x_0}{\alpha} = \frac{y - y_0}{\beta} = \frac{z - z_0}{\gamma}$.\\\\

Уравнение плоскости можно построить по трём точкам, зафиксируем первую точку, от неё проведём 2 вектора к двум оставшимся, теперь возьмём какую-то точку $(x, y, z)$, вектор от первой точки к новой должен быть ЛНЗ. Получили 3 вектора, которые образовали плоскость. Объём, натянутый на эти 3 вектора, равен 0, тогда мы просто пишем объём через определитель, раскрываем, получаем: $Ax + By +Cz +D = 0$.\\

Давайте до кучи напишем параметрическое уравнение плоскости:
\[
\begin{pmatrix}
x \\
y \\
z
\end{pmatrix}
=
\begin{pmatrix}
x_0 \\
y_0 \\
z_0
\end{pmatrix}
+
\lambda
\begin{pmatrix}
u_1 \\
u_2 \\
u_3
\end{pmatrix}
+
\mu
\begin{pmatrix}
v_1 \\
v_2 \\
v_3
\end{pmatrix}
\]

где u и v - базисные векторы плоскости, а $(x_0, y_0, z_0)$ - какая-то начальная точка.
\end{remarkbox}

\begin{theorembox}
Формула перехода к новому базису.\\

\textbf{Доказательство:}

Рассмотрим вектор $(x, y)$ в базисе $\{e_1, e_2\}$. Тогда наш вектор $a = xe_1 + ye_2$, а в другом базисе наш вектор - это $a = x^`e^`_1 + y^`e^`_2$. Хотим узнать $(x^`_1, y^`_2)$. $e^`_1 = a_{11}e_1 + a_{21}e_2$ и $e^`_2 = a_{12}e_1 + a_{22}e_2$. Подставим и преобразуем, потом воспользуемся единственностью представления вектора в данном базисе, тогда $x = x^`a_{11} + y^`a_{12}$, аналогично для y.
\end{theorembox}

\begin{remarkbox}
Вопрос исследования взаимного расположения прямой и плоскости, плоскости и плоскости довольно тривиальный, просто пользуемся параметрическим уравнением плоскости и прямой, а дальше очев: сводим задачу к исследованию расположения направляющего вектора прямой и нормаль-вектора плоскости и тд и тп, просто системы уравнений.
\end{remarkbox}

\begin{definitionbox}
\textbf{Угол между прямой и плоскостью} - это угол между прямой и проекцией прямой на данную плоскость.
\end{definitionbox}

\begin{theorembox}
Формула расстояния от точки $(x_0, y_0, z_0)$ до плоскости $\alpha: Ax + By + Cz + D = 0$ - это $d(x_0, y_0, z_0) = \frac{|Ax_0 + By_0 + Cz_0 + D)|}{\sqrt{A^2 + B^2 + C^2}}$\\

\textbf{Доказательство:}

Выводится также, как и для двумерного случая.
\end{theorembox}

\begin{remarkbox}
Расстояние между плоскостями или прямыми, или прямыми и плоскостями - это длина общего перпендикуляра. Задача сводится к расстоянию от точки до плоскости/прямой.
\end{remarkbox}

\begin{homeworkbox}
    Из Смирнова решайте все задачи из 3.1-3.4, 4.1-4.4 и 474-482. Не обяз решать всё, решайте пока не почувствуете, что прониклись. Задачи халява, все идеи есть в лекциях.
\end{homeworkbox}

\section{Векторные пространства и матрицы}

\begin{definitionbox}
    Пусть у нас есть векторное пространство размерности n, тогда набор векторов $\{e_1, \dots, e_n\}$ будет называться \textbf{базисом} этого пространства, если они все линейно независимы: $\sum \lambda_i \cdot e_i = 0 \Rightarrow \lambda_i = 0$.
\end{definitionbox}

\begin{theorembox}
У любого конечномерного векторного пространства сущестувет базис.\\

\textbf{Доказательство:}

Просто предъявляем алгоритм. Возьмём $e_1 = v_1 \neq 0$, потом возьмём $e_2 = v_2 \in V$ такой, что $\lambda_1 \cdot e_1 + \lambda_2 \cdot e_2 = 0 \Rightarrow \lambda_1 = 0, \lambda_2 = 0$. Если такой не нашёлся, тогда у нас базис из 1 вектора, имеем просто одномернопространство. Потом возьмём $e_3 = v_3 \in V \ \dots$ и тд. 
\end{theorembox}

\begin{remarkbox}
Система векторов обладает единственным базисом только в случае 0-мерного пространства.
\end{remarkbox}

\begin{definitionbox}
    Векторное пространство W представимо в виде \textbf{прямой суммы} пространств U и V, если $\forall w \in W \ \exists u \in U \ \& \ \exists v \in V \ | \ w = u + v$. Очень важно, что U и V пересекаются только по нулевому вектору.
\end{definitionbox}

\begin{theorembox}
a) Размерность подпространства не превосходит размерности пространства.\\
б) W - векторное пространство, U - его подпространство, тогда $\exists V$ такое, что $W = U + V$.\\

\textbf{Доказательство:}

а) очевидно, пытливый читатеть может самостоятельно привести доказательство этого пункта.\\
б) Выбираем базис в U, потом дополняем его до всего базиса W, тогда $\forall w \in W \ w = \sum_{i = 1}^{k} w_i \cdot e_i + \sum_{i = k + 1}^{n} w_i \cdot e_i$, где $k = dimU, n = dimV$. Тогда первая сумма у нас лежит в U, а вот то, что осталось мы определим как V, тогда базис нового пространства - это просто те, вектора, которыми мы дополнили базис U до базиса W. 
\end{theorembox}

\begin{remarkbox}
Если V - в.п. ($dimV = n$) над полем из q элементов, тогда всего векторов у нас $q^n$ (потому что $v = \sum q_i \cdot v_i$), а способов выбрать базис - $(q^n - 1)(q^n - q)\dots(q^n - q^{n - 1})$ (первым берём любой \textbf{ненулевой} вектор, потом берем вектор, который ЛНЗ с первым, то есть $e_2 \neq \lambda_1\cdot e_1$, на место лямбды q вариантов и тд).
\end{remarkbox}

\begin{theorembox}
Ранг матрицы $A|B$ (\textit{это приписывание матрицы B справа от матрицы A}) не превосходит суммы рангов матриц A и B.\\

\textbf{Доказательство:}

$A|B = A|0 + 0|B$, тогда $rk(A|B) = rk((A|0) + (0|B)) \leq rk(A|0) + rk(0|B) = rk(A) + rk(B)$.
\end{theorembox}

\begin{theorembox}
Всякую матрицу ранга r можно представить в виде суммы r матриц ранга 1, но нельзя представить в виде суммы меньшего числа таких матриц.\\

\textbf{Доказательство:}

$rkA = r, A = \sum A^`_i$. Без ограничения общности будем считать, что у нас ЛНЗ первые r строчек матрицы:

\[
\begin{pmatrix}
A_1 \\
A_2 \\
\dots \\
A_r\\
A_{r + 1}\\
\dots\\
A_n
\end{pmatrix}
=
\begin{pmatrix}
A_1 \\
0 \\
\dots \\
0\\
\lambda_1^{r+1} A_1\\
\lambda_1^{r+2} A_1\\
\dots
\end{pmatrix}
+
\begin{pmatrix}
0 \\
A_2 \\
\dots \\
0\\
\lambda_2^{r+1} A_2\\
\lambda_2^{r+2} A_2\\
\dots
\end{pmatrix}
+
\dots
\]

$rkA = rk(\sum A_k) \leq \sum rk(A_k) = k$, если k < r, тогда $rkA < r$ - противоречие.
\end{theorembox}

\begin{theorembox}
$A^T A = A A^T \Rightarrow (A^{-1})^T = A^{-1}$.\\

\textbf{Доказательство:}

$A^{-1}A = E$, транспонируем $(A^{-1}A)^T = E^T \Leftrightarrow A^T (A^{-1})^T = E^T$, домножим слева на $A^{-1}$, получим $(A^{-1})^T = A^{-1}$
\end{theorembox}

Далее в лекции разобраны несколько опорных задач, связанных с коммутативностью и обратимостью матрицы. Доказательства как правило проводились через \textbf{след матрицы}, потому что работать с числами куда приятнее и понятнее, чем с матрицами. Не считаю нужным конспетировать эти задачи. Может быть кто-то захочет продолжить моё дело и откроет пул реквест.

\begin{definitionbox}
    \textbf{Следом квадратной матрицы} A (dimA = n) мы будем называть $tr(A) = \sum_{i = 1}^n a_{ii}$.\\

    \textbf{След} становится удобным инструментом в доказательстве теорем про матрицы благодаря ряду свойств:
    \begin{enumerate}
        \item $tr(\alpha A + \beta B) = \alpha \cdot tr(A) + \beta \cdot tr(B)$
        \item $tr(C^{-1}AC) = tr(A)$, в частности $tr(AB) = tr(BA)$
        \item $tr(A) = tr(A^T)$
        \item \textbf{След матрицы} равен сумме её собственных значений.
    \end{enumerate}
\end{definitionbox}

\begin{remarkbox}
    Строковый и столбцовый ранг совпадают.
\end{remarkbox}

\begin{definitionbox}
    Симметрические матрицы: $A = A^T$, кососимметрические матрицы: $A^T = -A$. Также заметим, что в последнем случае у нас обязательно на диагонали должны быть нули, т.к. $a_{ii} = -a_{ii} \Leftrightarrow a_{ii} = 0$, а остальные элементы $a_{ij} = -a_{ji}$.
\end{definitionbox}

\begin{remarkbox}
    Рассмотрми пространства симметрических матриц (U) и кососимметрических матриц (V). Размерность первого пространства - это $\frac{n^2 + n}{2}$ (потому что такая матрицы задаётся с помощью n чисел на диагонали + количество чисел над диагональю, для этого нужно из всех чисел матрицы вычесть диагональ и поделить пополам - $\frac{n^2 - n}{2}$. Размерность второго пространства тогда - это $\frac{n^2 - n}{2}$ (раз на диагонали только нули).\\

    Теперь сложим эти размерности $\frac{n^2 + n}{2} + \frac{n^2 - n}{2} = \frac{2n^2}{2} = n^2$. Получили размерность всего пространства квадратных матриц $M_n(\R)$. Также заметим, что пространства симметрических и кососимметрических матриц пересекаются только по нулевой матрице (то есть $A = - A$). Тогда $A \in M_n(\R) \ \Rightarrow \ \exists U, V \ A = U \oplus V$ причём U - симметрическая матрица, а V - кососимметрическая матрица.
\end{remarkbox}

\begin{theorembox}
    $rk(A + B) \leq rk(A) + rk(B)$\\

    \textbf{Доказательство:}

    Идейно: $rk(A+B) = rk(A) +rk(B) - rk(A\cap B) \leq rk(A) + rk(B)$
\end{theorembox}

\begin{remarkbox}
    $tr(AB) = tr(BA)$, но $rk(AB) \neq rk(BA)$
\end{remarkbox}

\begin{definitionbox}
    $\R_n[x]$ - пространство многочленов, базисом которого может быть, например, $\{1, x, x^2,\dots, x^n\}$.\\

    Скалярное произведение двух функций $f$ и $g$ - это $(f, g) = \int_a^b f(x)g(x)dx$.
\end{definitionbox}

\begin{remarkbox}
    А скалярное произведение матриц - это tr их произведения. Можете прогнать по свойствам скалярного произведения и убедиться в этом.
\end{remarkbox}

\begin{theorembox}
    Рассмотрим набор векторов $\{e_1, e_2, \dots,e_n\}$ со следующим свойством:
    \[
        \begin{cases}
        (e_i, e_i) = 1\\
        (e_i, e_j) = 0
        \end{cases}
    \]
    Докажем, что этот набор векторов является базисом.\\

    \textbf{Доказательство:}

    Проверим ЛНЗ. Хотим $\sum \lambda_i e_i = 0 \Rightarrow \forall i \ \lambda_i = 0$. Рассмотрим скалярное произведение $\forall i \ (e_i, \sum \lambda_i e_i) =0 $, раскроем по свойству линейности, получим что-то такое: $\sum \lambda_j (e_i, e_j) = \lambda_i (e_i, e_i) = 0 \Rightarrow \lambda_i = 0$.
    
\end{theorembox}

\begin{definitionbox}
    Набор векторов из теоремы выше называется \textbf{ортонормированным базисом}.
\end{definitionbox}

\begin{definitionbox}
    $W^\perp := \{w^\perp \in W^\perp \ | \ \forall w \in W \ (w^\perp, w) = 0\}$
\end{definitionbox}

\begin{remarkbox}
    $V = W \oplus W^\perp$. Очевидно, что $W$ и $W^\perp$ пересекаются только по нулю, если бы мы нашли какой-то $x \in W, W^\perp$, то получили бы что-то в духе $(x, x) = 0$, а отсюда по свойству скалярного произведения получаем, что $x = 0$.
\end{remarkbox}

\begin{theorembox}
    Метод Грама–Шмидта: Любой базис $ \{e_1, ..., e_n\} $ евклидова пространства можно преобразовать в ортонормированный базис $ \{f_1, ..., f_n\} $ следующим образом:

    \begin{align*}
    u_1 &= e_1, \\
    u_2 &= e_2 - \frac{(e_2, u_1)}{(u_1, u_1)} u_1, \\
    u_3 &= e_3 - \frac{(e_3, u_1)}{(u_1, u_1)} u_1 - \frac{(e_3, u_2)}{(u_2, u_2)} u_2, \\
    &\vdots \\
    f_i &= \frac{u_i}{\|u_i\|}.
    \end{align*}
\end{theorembox}

\begin{theorembox}
    Пусть A - матрица размера $n \times n$. Если для любой матрицы $X$ размера $n \times n$ справедливо равенство $tr(AX) = 0$, то $A = 0$.\\

    \textbf{Доказательство:}

    Попробуем $X = A^T$. $tr(AA^T) = \sum a_{ij}^2 = 0 \ \Rightarrow \ \forall i, j \ a_{ij} = 0$. Либо можно сказать, что у нас $tr(AX)$ - это скалярное произведение на пространстве матриц, причём у нас $\forall X \ tr(AX) = 0$, то есть A перпендикулярно любому вектору, а это возможно в том случае, если A - это нулевой вектор.
    
\end{theorembox}

\begin{homeworkbox}
    Домашка есть в \texttt{Векторные пространства 102.pdf}
\end{homeworkbox}

\end{document}
